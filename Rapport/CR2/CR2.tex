\documentclass[11pt]{meetingmins}
    \usepackage[utf8]{inputenc}   
    \usepackage[T1]{fontenc}
    \usepackage{tabularx}
    \newcolumntype{C}{>{\centering}X}
    \usepackage{colortbl}
    
    %% Optionally, the following text could be set in the file
    %% department.min in this folder, then add the option 'department'
    %% in the \documentclass line of this .tex file:
    %%\setcommittee{Department of Instruction}
    %%
    %%\setmembers{
    %%  \chair{B.~Smart},
    %%  B.~Brave,
    %%  D.~Claire,
    %%  B.~Gone
    %%}
    
    \setcommittee{Projet de
    compilation : Compte-rendu de réunion 1}
    
    
    \setmembers{
     Alexandre GIGLEUX, 
      Ali LABBAIZE , 
      Guillaume GATTI, 
      Florian RENOU
    }
    
    \setdate{Octobre 24, 2018}
    
    \setpresent{
     Alexandre GIGLEUX, 
      Ali LABBAIZE , 
      Guillaume GATTI, 
      Florian RENOU
    }
    \absent{ Aucun }
    \begin{document}
    \maketitle
    \section{Type de réunion : \textnormal{Réunion de lancement de projet et première répartition des tâches} }
    
    \section{Durée : \textnormal{50min} }
    \section{Lieu : \textnormal{ Salle 1.11} }
    
    
    \section{}
    \section{Ordre du jour :}
    \begin{hiddenitems}
    \item
    Se familiariser avec le language Tiger à travers la documentation
    \item
    Vérifications si la grammaire est LL(1)
    \item
    Traduction de la grammaire vers un fichier.g (Antlr) 
    \item
    Présentation du GANTT
    \item
    \end{hiddenitems}
    
    \section{Informations échangées :}
    \begin{hiddenitems}
    
    \item Explication des points de difficultés à propos de la grammaire notommant pour la création de type équivalent à des
    structures en language C par rapport à la déclaration de var
    
    \item
    Échange des informations récoltées sur le sujet:
    \begin{itemize}
        \item \textbf{A.~LABBAIZE : } 
        \begin{itemize}
            \item Explication sur le type Record analogie avec les structures en C.
            \item Présentation du GANTT prévisionnel : Discution + remodification en fct des disponibilités des membres.
            \item Travail sur la grammaire (Premiers + Suivants)
        \end{itemize}
        
        \item \textbf{F. RENOU : }intérêt de déjà copier la grammaire en .g afin de la vérifier sur Antlr avant les vérifications manuelles.        
        
        \item \textbf{A. GIGLEUX : }
        \begin{itemize}
            \item Discute à propos du boucle dans la grammaire LL(1): au niveau de lValue. 
            \item Travail sur la grammaire (Premiers + Suivants)
        \end{itemize}
        
        \item \textbf{G. GATTI : }A réexpliquer pour tout le monde les étapes du TP qu'il a retravaillé chez lui.
        
    \end{itemize}
    
    
    \item Travail pour la semaine prochaine : 
    \begin{itemize}
        \item Symboles directeurs
        \item Diagramme de PERT + modification du GANTT
        \item Essayer de compiler la grammaire.g sur Antlr + modifications si nécessaire
    \end{itemize}
    
    
    \end{hiddenitems}
    \section{Décisions : }
    \begin{itemize}
        \item
        Choix de commencer en parallèle la création du fichier.g lié à la grammaire ET étude manuelle de la grammaire(Premiers + Suivants)
        \item
        Choix du template pour la rédaction du rapport .
        \item
        Mise en place du Git + Création de l'architecture du projet.
        \item
        Election du chef de projet : Ali LABBAIZE.
        \item La grammaire sera en anglais
        \item Construire un diagramme GANTT/PERT
        \item Faire une réunion de chantier chaque semaine afin d'éviter l'effet tunnel
        \end{itemize}

    \section{Todo list :}
    \begin{table}[h]
        \centering
        \begin{tabular}{|p{4cm}|c|c|c|c|}
            \hline
            \rowcolor{yellow} Description & Responsable & Délai & Livrable & Validé par
            \tabularnewline \hline
            Etat d'art & A.LABBAIZE. & Pour le 30/03/2018 & Fichier LaTex & Toute l'équipe
            \tabularnewline \hline
            Se renseigner sur les interfaces graphiques & R. BANEL & Pour le 30/03/2018 & - & Toute l'équipe
            \tabularnewline \hline
            Réalisation du GANTT  & AL/PLF & Pour le 30/03/2018 & Diagramme & Toute l'équipe 
            \tabularnewline
            \hline  
        \end{tabular}
        \caption{Distribution des tâches}
        \label{tab:my_label}
    \end{table}
    
    \section{Questions/Remarques :}
    \begin{itemize}
        \item Est ce qu'on doit faire un état de l'art ?
    \end{itemize}
    \section{Date de la prochaine réunion : \textnormal{le 24 Mars 2018} }
    
    
    
    \end{document}