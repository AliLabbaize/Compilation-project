\documentclass[11pt]{meetingmins}
    \usepackage[utf8]{inputenc}   
    \usepackage[T1]{fontenc}
    \usepackage{tabularx}
    \newcolumntype{C}{>{\centering}X}
    \usepackage{colortbl}
    
    %% Optionally, the following text could be set in the file
    %% department.min in this folder, then add the option 'department'
    %% in the \documentclass line of this .tex file:
    %%\setcommittee{Department of Instruction}
    %%
    %%\setmembers{
    %%  \chair{B.~Smart},
    %%  B.~Brave,
    %%  D.~Claire,
    %%  B.~Gone
    %%}
    
    \setcommittee{Projet de
    compilation : Compte-rendu de réunion 1}
    
    
    \setmembers{
     Alexandre GIGLEUX, 
      Ali LABBAIZE , 
      Guillaume GATTI, 
      Florian RENOU
    }
    
    \setdate{Octobre 10, 2018}
    
    \setpresent{
     Alexandre GIGLEUX, 
      Ali LABBAIZE , 
      Guillaume GATTI, 
      Florian RENOU
    }
    \absent{ Aucun }
    \begin{document}
    \maketitle
    \section{Type de réunion : \textnormal{Réunion de lancement de projet et première répartition des tâches} }
    
    \section{Durée : \textnormal{50min} }
    \section{Lieu : \textnormal{ Salle 1.11} }
    
    
    \section{}
    \section{Ordre du jour :}
    \begin{hiddenitems}
    \item
    Partage des points de vue et des connaissances récoltées sur le sujet.
    \item
    Répartition des tâches en fonction des compétences de chacun.
    \item
    Se mettre d'accord sur l'environnement de travail (Code block, éditeur de texte ..etc)
    \item
    La mise en évidence des risques temporels. Définir les étapes du projet, afin de mettre en place un GANTT.
    \item
    Création du Git
    \item
    Election du responsable de projet.
    \end{hiddenitems}
    
    \section{Informations échangées :}
    \begin{hiddenitems}
    
    \item Faire une réunion de chantier chaque semaine afin d'éviter l'effet tunnel.
    
    \item
    Échange des informations récoltées sur le sujet:
    \begin{itemize}
        \item \textbf{A.~LABBAIZE : }propose de mettre en place un système de notation qui se base sur 2 approches, une notation par critère (genre du film, acteur, réalisateur,...) et une notation basée sur le choix des utilisateurs. L'idée serait de créer un système se basant sur ces 2 notations.
        
        
    
        \item \textbf{R. BANEL : }parle de son expérience sur les projets en tant que R1. Trois grands conseils en sont ressortis :
        \begin{itemize}
            \item La mise en œuvre des outils de gestion de projet est primordiale
            \item Les tests sont obligatoires
            \item Il faut tout expliciter dans le rapport final, même ce qu'on va remplacer ou effacer
        \end{itemize}
        
        
        \item \textbf{PL. FEULVARCH : }explique que pour gagner en complexité, il serait possible d'utiliser des systèmes de recommandation prédéfinis.
        
    \end{itemize}
    
    
    \item Travail pour la semaine prochaine : 
    \begin{itemize}
        \item Définir les lots de travail 
        \item Faire un diagramme de Gantt, puis un diagramme de PERT
        \item Commencer la verifications de la grammaire
        \item Prendre en main ANTLR + debut fichier.g
        \item Relecture des spécifications de la grammaire
    \end{itemize}
    
    
    \end{hiddenitems}
    \section{Décisions : }
    \begin{itemize}
        \item
        Choix de réunion hebdomadaire : le mercredi de 14h à 16h .
        \item
        Choix du template pour la rédaction du rapport .
        \item
        Mise en place du Git + Création de l'architecture du projet.
        \item
        Election du chef de projet : Ali LABBAIZE.
        \item La grammaire sera en anglais
        \item Construire un diagramme GANTT/PERT
        \item Faire une réunion de chantier chaque semaine afin d'éviter l'effet tunnel
        \end{itemize}

        
    \section{Todo list :}
    \begin{table}[h]
        \centering
        \begin{tabular}{|p{4cm}|c|c|c|c|}
            \hline
            \rowcolor{yellow} Description & Responsable & Délai & Livrable & Validé par
            \tabularnewline \hline
            Etat d'art & A.LABBAIZE. & Pour le 30/03/2018 & Fichier LaTex & Toute l'équipe
            \tabularnewline \hline
            Se renseigner sur les interfaces graphiques & R. BANEL & Pour le 30/03/2018 & - & Toute l'équipe
            \tabularnewline \hline
            Réalisation du GANTT  & AL/PLF & Pour le 30/03/2018 & Diagramme & Toute l'équipe 
            \tabularnewline
            \hline  
        \end{tabular}
        \caption{Distribution des tâches}
        \label{tab:my_label}
    \end{table}
    
    \section{Questions/Remarques :}
    \begin{itemize}
        \item Est ce qu'on doit faire un état de l'art ?
    \end{itemize}
    \section{Date de la prochaine réunion : \textnormal{le 24 Mars 2018} }
    
    
    
    \end{document}